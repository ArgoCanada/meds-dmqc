% !TEX encoding = UTF-8 Unicode

\documentclass[11pt,english]{article} % document type and language

\usepackage{babel}    % multi-language support
\usepackage{float}    % floats
\usepackage{url}      % urls
\usepackage{graphicx} % figures

\title{Plan for BGC DMQC at the MEDS DAC}
\author{Christopher Gordon\\Fisheries \& Oceans Canada}
\date{\today}

\begin{document}

\maketitle

\section{Introduction}

As the \verb|bgcArgoDMQC| package reaches an operational stage of development,
it is time to begin performing DMQC on the DOXY floats where possible. Taking
into consideration the current data mode state of profiles on the MEDS DAC and
what our largest priorities are, I have a few possible ``strategies'' for doing
so presented in section \ref{s:plan}.

\section{Current Status}

The quality flags of TEMP and PSAL affect the quality flags of DOXY, so it is 
important for the Core DMQC process to be complete before performing the BGC
DMQC process. The current data modes of profiles for the MEDS DAC are:

\begin{table}[H]
    \centering
    \caption{Data mode for DOXY and Core variables}
    \begin{tabular}{r|ll}
        \hline
        Data Mode & \# Profiles & Percent \\
        \hline\hline
        Both RT & 390 & 8.5\% \\
        BGC RT, Core DM & 825 & 18.0\% \\
        BGC A, Core RT & 703 & 15.3\% \\
        BGC A, Core DM & 2046 & 44.6\% \\
        Both DM & 619 & 13.5\% \\
        \hline
    \end{tabular}
\end{table}

Excluding any profiles less than 1~yr old shifts the ditribution slightly, 
mostly by reducing the number of Core profiles in RT mode: 

\begin{table}[H]
    \centering
    \caption{Data mode for DOXY and Core variables for profiles older than 1~yr}
    \begin{tabular}{r|ll}
        \hline
        Data Mode & \# Profiles & Percent \\
        \hline\hline
        Both RT & 240 & 9.0\% \\
        BGC RT, Core DM & 725 & 19.0\% \\
        BGC A, Core RT & 701 & 16.2\% \\
        BGC A, Core DM & 2046 & 47.1\% \\
        Both DM & 619 & 14.3\% \\
        \hline
    \end{tabular}
\end{table}

Looking at the data mode report generated by Henry Bittig, we can see where we
began putting DOXY floats into data mode ``A'', I assume using the DOXY audit
generated by Josh Plant. 

\begin{figure}[H]
    \centering
    \includegraphics[width=0.75\textwidth]{/Users/gordonc/Documents/argo/meds/bio_index_RAD_DOXY_evolution_meds.jpg}
    \caption{Data mode status of DOXY from report generated by Henry Bittig}
\end{figure}

\section{Priorities and Timeline} \label{s:plan}

Two plan options:

\begin{enumerate}

    \item[1 (a)] Simply begin with the oldest profile, work on all possible
    profiles for that float, and then move to the oldest profile, and repeat.
    \item[1 (b)] As above, but first prioritize floats that are no longer
    operational. This will move entire floats into delayed mode, and also
    will not merit any change to RT processing\footnote{When DM files exist
    for a DOXY float, a temporal extrapolation of the gain is supposed to
    be applied in RT}
    \item[2 (a)] Prioritize floats that already have adjustments made to them. 
    \item[2 (b)] The opposite of above, prioritize floats that are in RT mode
    and \emph{not} in A mode. Floats that already have adjusted data need DM
    processing less than those in RT mode, so more value is added by processing
    those floats first.

\end{enumerate}

As for the timeline, I believe that things will move slowly at first, as I am
likely to run into issues for the first few floats or profiles I process. The
process will speed up as I add ways to optimize my workflow into the python 
package. 

I think a good and achievable goal would be to be ``caught up'' and have all
required ($>$ 1~yr old) floats DMQC'ed by the next ADMT meeting. 

\end{document}
